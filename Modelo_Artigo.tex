\pdfcompresslevel 9
\documentclass[conference]{ModeloA}
\usepackage[portuguese]{babel}
\usepackage{cite}
\usepackage{amsmath,amssymb,amsfonts}
\usepackage{algorithmic}
\usepackage{graphicx}
\usepackage{textcomp}
\usepackage{xcolor}
\usepackage{hyperref}
\def\BibTeX{{\rm B\kern-.05em{\sc i\kern-.025em b}\kern-.08em
    T\kern-.1667em\lower.7ex\hbox{E}\kern-.125emX}}
\usepackage[all]{hypcap}
\hypersetup{    
	colorlinks=true,    
	linkcolor=blue,    
	urlcolor=blue,  
	citecolor=red,  
	linktoc=all}

\begin{document}

\title{Como veio ao mundo jogos que não tem pausa}

\author{
	\IEEEauthorblockN{Gabriel Rossine Paulino de Souza}
	\IEEEauthorblockA{\textit{ECT - UFRN} \\
		Parnamirim, Rio Grande do Norte, Brasil \\
		girossine123@gmail.com }
\and
	\IEEEauthorblockN{Pedro Arcanjo Mendes de Morais}
	\IEEEauthorblockA{\textit{ECT - UFRN} \\
		Parnamirim, Rio Grande do Norte, Brasil \\
		pedro.arcanjo.mendes@gmail.com }
}


\maketitle
%-------------------------------------------------------------------------
% RESUMO
\begin{abstract}
	Neste trabalho abordaremos a evolução, importância e o impacto dos jogos onlines na sociedade moderna. Serão discutidos neste trabalho a história do mundo multiplayer dentre as tecnologias e o crescimento em meio as pessoas.
\end{abstract}

%-------------------------------------------------------------------------
% Palavras chave
\begin{IEEEkeywords}
tecnologia, jogos, multiplayer-online, surgimento
\end{IEEEkeywords}

%-------------------------------------------------------------------------
% Introdução

\section{Introdução}
Jogos Online estão se tornando cada vez mais uma febre para as novas gerações, onde ao percorrer dos anos está se expandindo entre todas as áreas de conhecimento como por exemplos: econômica, tecnológica e cultural. E este trabalho tem como o objetivo adentrar este mundo que cresce desenfreadamente destacando seu desenvolvimento e importância desses eventos que o fizeram no cenário atual.

%-------------------------------------------------------------------------
% Exigências

\section{ARPANET}
As raízes dos jogos online estão em algumas das primeiras tecnologias de rede de computadores. No final de 1970, muitas universidades americanas dos Estados Unidos foram conectadas pela ARPANET, um precursor da internet. A ARPANET (Em inglês, Advanced Research Project Agency Network, ou em potuguês, Rede da Agência para Projetos de Pesquisa Avançada) foi uma rede de computadores desenvolvida em 1969 para transmitir dados militares sigilosos nos Estados Unidos. Pode-se considerar a ARPANET como a mãe da internet, pois foi a primeira a implementar protocolos TCP/IP, que são a base de muitos outro protocolos utilizados na internet nos tempos atuais.
\cite{WikipediaArpanet}
\section{Multiuser Dungeon o primeiro}
A ARPANET permitia usuários conectarem seus computadores a um computador central e interagir em um tempo de resposta que era perto de tempo real (Imediato). Em 1980 ARPANET foi conectado a Universidade de Essex onde dois estudantes criaram um jogo de aventura e fantasia baseado em texto online chamado Multiuser Dungeon (MUD) conectado através da ARPANET. Quando o primeiro usuário de fora conectou a MUD, jogar jogos online nasceu para os jogadores.
\cite{Britannica}
\section{Jogos que revolucionaram o seu tempo}
No início da década de 1990, tivemos depois de alguns anos o jogo“NeverWinter Nights” (1991), um RPG de turnos em pixel art que permitia até 50 jogadores online. Foi um grande avanço para os jogos onlines tendo um avanço de tecnologia onde saía de jogos de escrever para jogos de rpg de turnos bem simples  . Com a grande evolução das tecnologias e da internet, jogos como "Doom" (1993) e "Quake" (1996) permitiram partidas multiplayer em rede, popularizando o conceito de jogos cooperativos e competitivos online. Esses jogos hospedavam servidores locais e foram a base para a criação de comunidades de jogadores.
\cite{WikipediaNWN}
\section{O crescimento da comunidade}
O grande avanço desse mercado ocorreu nos anos 2000, com o surgimento de diversos gêneros, incluindo jogos de tiro, simulação e, de forma muito significativa, os MMORPGs (Massively Multiplayer Online Role-Playing Games). Um dos mais importantes dessa categoria foi "World of Warcraft" ou WOW (2004), que se tornou um ícone no mundo dos jogos online, oferecendo um mundo persistente onde milhares de jogadores podiam interagir simultaneamente, beneficiando-se do avanço da banda larga na internet. Outros jogos notáveis dessa época incluem "Counter-Strike 1.6" ou CS (2000) e "The Sims Online" (2002). "Counter-Strike 1.6" foi pioneiro como jogo de tiro online em primeira pessoa, enquanto "The Sims Online" explorava a simulação de vida em comunidade. Na década de 2010, a tecnologia e a internet continuaram a crescer, levando ao surgimento de novos jogos e comunidades. A economia dos jogos online e sua influência cultural se expandiram significativamente graças à internet. Essa trajetória de evolução destaca o impacto duradouro e crescente dos jogos online desde os anos 1970 até hoje.
\cite{WikipediaNWN}

%-------------------------------------------------------------------------
% Conclusão

\section{Conclusão}

Neste trabalho vimos a evolução dos jogos onlines que foi permitida pelo avanço da tecnologia da comunicação. Destaca-se uma tecnologia criada para propósitos militares que acabou tornando-se de uso geral, a ARPANET. Essa tecnologia possibilitou a entrada dos jogos nas redes de computadores, conhecida hoje como internet. O jogo MUD foi a entrada para o mundo dos jogos online, DOOM e QUAKE revolucionou os gráficos 3D de sua época, outros jogos citados como WOW, CS e The Sims Online fizeram crescer cada vez mais as comunidades de jogadores online. Esse grande crescimento de jogadores acabou permitindo que esses jogos se sustentassem e evoluíssem para o que são hoje.

\begin{thebibliography}{00}
	
\bibitem{WikipediaArpanet} Wikipedia - ARPANET. {\it Disponível em \href{https://pt.wikipedia.org/wiki/ARPANET}{ARPANET - Wikipedia}}

\bibitem{Britannica} Britannica - Online gaming. {\it Disponível em \href{https://www.britannica.com/technology/online-gaming}{Online gaming History, Examples, Companies, & Facts - Britannica}}

\bibitem{WikipediaNWN} Britannica - Online gaming. {\it Disponível em \href{https://en.wikipedia.org/wiki/Neverwinter_Nights_(1991_video_game)}{Neverwinter Nights (videogame de 1991) – Wikipédia}}

\end{thebibliography}


\end{document}
