\pdfcompresslevel 9
\documentclass[conference]{ModeloA}
\usepackage[portuguese]{babel}
\usepackage{cite}
\usepackage{amsmath,amssymb,amsfonts}
\usepackage{algorithmic}
\usepackage{graphicx}
\usepackage{textcomp}
\usepackage{xcolor}
\usepackage{hyperref}
\def\BibTeX{{\rm B\kern-.05em{\sc i\kern-.025em b}\kern-.08em
    T\kern-.1667em\lower.7ex\hbox{E}\kern-.125emX}}
\usepackage[all]{hypcap}
\hypersetup{    
	colorlinks=true,    
	linkcolor=blue,    
	urlcolor=blue,  
	citecolor=red,  
	linktoc=all}

\begin{document}

\title{Como veio ao mundo jogos que não tem pausa}

\author{
	\IEEEauthorblockN{Gabriel Rossine Paulino de S.}
	\IEEEauthorblockA{\textit{ECT - UFRN} \\
		Cidade, estado, País \\
		email }
\and
	\IEEEauthorblockN{Pedro Arcanjo Mendez}
	\IEEEauthorblockA{\textit{ECT - UFRN} \\
		Cidade, estado, País \\
		email }
}


\maketitle
%-------------------------------------------------------------------------
% RESUMO
\begin{abstract}

Este documento traz recomendações para a escrita de trabalho no formato de artigo científico na linguagem \LaTeX. O documento define formas e itens necessários ao trabalho. 

No resumo e no título do artigo não é indicado o uso de símbolos, caracteres especiais, notas de rodapé ou equações matemáticas.

\end{abstract}

%-------------------------------------------------------------------------
% Palavras chave
\begin{IEEEkeywords}
tecnologia, jogos, multiplayer-online, surgimento
\end{IEEEkeywords}

%-------------------------------------------------------------------------
% Introdução

\section{Introdução}
Este documento é um modelo com as instruções para a construção da atividade 2 do componente curricular ``Introdução às Ciências e Tecnologia'' no semestre letivo 2024.1, sob a coordenação da professora Jossana Ferreira. 

O trabalho deve ser fruto de uma pesquisa científica e não uma cópia de fontes de pesquisa, os autores devem escrever com suas próprias palavras, de forma clara e objetiva.

%-------------------------------------------------------------------------
% Exigências

\section{Exigências do trabalho}

Nesta seção serão mostrados os itens obrigatórios e a forma como devem ser apresentados.

\subsection{Número de participantes}
 O trabalho será desenvolvido em duplas ou individualmente, e as duplas devem ser indicadas na planilha do endereço:\\ 
  \href{https://docs.google.com/spreadsheets/d/1l2WnmjrgzaTikbDhNdt9FAAWOUk1IO63smh_fKgjOUs/edit?usp=sharing}{LINK DAS DUPLAS}\\ 
 para que possam ser cadastradas no SIGAA.

\subsection{Tema do trabalho}

O tema do trabalho deve ser definido pela dupla. Devem escolher um produto, uma tecnologia ou um processo e explicar sua origem, evolução, situação atual e perspectiva de desenvolvimento para o futuro. Como exemplo, pode ser citado o telefone celular, o processo de filtragem para obtenção de água própria para consumo, os satélites espaciais, os CubeSats, os microchips, a nanotecnologia, a mecânica quântica, os motores elétricos, as viagens espaciais, a previsão de chuvas, etc.

\subsection{Entrega do trabalho}

O trabalho deve ser entregue no formato PDF, através da plataforma SIGAA, por um dos membros da dupla até a segunda feira dia 15 de julho de 2024.

%-------------------------------------------------------------------------
% Essenciais
	
\section{Seções essenciais}

O trabalho deve ter duas páginas (nem mais nem menos), em \LaTeX, no formato especificado no modelo.
Para o desenvolvimento do trabalho é necessário que hajam no mínimo as seções:\\

\begin{itemize}
	\item {\bf Resumo:} Apresentar um resumo objetivo do que será mostrado no trabalho.
	\item {\bf Introdução: }
			Contextualização, explicação do problema, importância do tema.
	\item {\bf Histórico:} 
			Resgate do nascimento e história do produto desde o seu início, com datas e motivação para sua criação.
	\item {\bf Seções de desenvolvimento:}
			Incluir quantas seções forem necessárias para explicar o seu tema.
	\item {\bf Conclusão:} 
			Incluir um resumo com o fechamento das ideias incluindo os principais pontos mostrados.
	\item {\bf Referências bibliográficas:}
			Incluir todas as fontes pesquisadas, sejam livros, sites, canais, apostilas, notas de aula, etc.
\end{itemize}

%-------------------------------------------------------------------------
% Textual

\section{Exigências textuais}

O trabalho deve ser desenvolvido em folha A4, fonte Times New Roman de tamanho 12pt e em duas colunas. As margens já são definidas na formatação do modelo.

Quando utilizar abreviações e acrônimos, defina o seu significado logo na sua primeira aparição no texto. Tente não utilizar essas simplificações no título ou resumo.

Se for necessário usar unidades, opte sempre pelo sistema internacional de unidades, não utilize padrões distintos, por extenso e depois abreviado, escolha uma forma e a mantenha em todas as utilizações (exemplo ``$Wb/m^2$'' ou ``webers por metro quadrado").

Use ``0'' antes do número decimal: exemplo 0,35.

Para citar uma referência basta fazer menção ao código usado na referência que a ordem será automaticamente definida, como por exemplo as referências bibliográficas \cite{Diniz},  \cite{Bazzo} e \cite{Faria} ou as referências digitais \cite{overleafFig} e \cite{overleafTab}.

%-------------------------------------------------------------------------
% Ferramentas

\section{Ferramentas para o artigo}

Nesta seção serão mostradas algumas ferramentas que podem ou não ser usadas no trabalho, são totalmente opcionais, mas podem ajudar a explicar melhor as ideias. No caso de utilização, todas as figuras, tabelas e equações devem ser mencionadas (citadas) no texto antes da sua aparição.

\subsection{Figura}
Para inserir uma figura basta localizar no texto na posição desejada e citá-la usando uma referência definida, dessa forma, ao adicionar outra figura, a numeração se adequa automaticamente, como por exemplo, a Figura \ref{ect}.

\begin{figure}[htbp]
	\centerline{\includegraphics[width=4cm]{ECT.png}}
	\caption{Examplo de Figura.}
	\label{ect}
\end{figure}

\subsection{Tabela}

As tabelas seguem a mesma flutuação das figuras, dois exemplos são mostrados nas Tabelas \ref{tab1} e \ref{tab2}.

\begin{table}[htbp]
	\caption{Exemplo de Tabela 1}
	\begin{center}
		\begin{tabular}{|l|c|c|}
			\hline
			\textbf{Praia} & \textbf{Estado} & \textbf{Cidade} \\ \hline
			Pipa & RN & Goianinha \\
			Porto de Galinhas & PE & Ipojuca\\
			Iracema & CE & Fortaleza\\
			\hline 
		\end{tabular}
		\label{tab1}
	\end{center}
\end{table}


\begin{table}[h]
	\caption{Exemplo de Tabela 2}
	\begin{center}
		\begin{tabular}{|c|c|c|c|}
			\hline
			\textbf{Brasil}&\multicolumn{3}{|c|}{\textbf{Frutas}} \\
			\cline{2-4} 
			\textbf{Região} & \textbf{\textit{Amarelas}}& \textbf{\textit{Verdes}}& \textbf{\textit{Vermelhas}} \\
			\hline
			Nordeste & Laranja$^{\mathrm{a}}$& Limão & Acerola  \\
			Sul & Pequi & Lichia & Morango \\ 
			\hline
			\multicolumn{4}{l}{$^{\mathrm{a}}$Exemplo de nota de rodapé.}
		\end{tabular}
		\label{tab2}
	\end{center}
\end{table}

\subsection{Equação}

Para o caso de uma única equação, tem-se o exemplo da Equação \ref{delta}:

\begin{equation}
	\Delta^2 = a^2 + b^2 \label{delta}
\end{equation}

No caso de várias equações consecutivas, tem-se as Equações \ref{eqGama} e \ref{eqOmega}:

\begin{eqnarray}
	a+b=\gamma \label{eqGama}\\
	\omega = \frac{b}{a} \label{eqOmega}
\end{eqnarray}


%-------------------------------------------------------------------------
% Conclusão

\section{Conclusão}

Na conclusão deve ser mostrado o que se pôde concluir a partir dos fatos estudados, um breve resumo dos resultados e um fechamento das ideias devem ser apresentados de forma clara e objetiva.

Lembre que o trabalho não deve apresentar plágio (além de antiético, é crime). 


\begin{thebibliography}{00}
	
\bibitem{Diniz} H. C. Pereira-Diniz, ``Ciência e tecnologia: origem, evolução e perspectiva'', Belo Horizonte: BDMG, 2011.

\bibitem{Bazzo} W.A. Bazzo,``Introdução à engenharia: conceitos, ferramentas e comportamentos'', $1^a$ Ed., Florianópolis: Ed. da UFSC, 2007.

\bibitem{Faria} R. M. Faria (coordenador),  ``Ciência, tecnologia e inovação para um Brasil competitivo'', $1^a$ ed - São Paulo: SBPC, 2011.

\bibitem{overleafFig} Site Overleaf - Inserindo imagens. {\it Disponível em \href{https://pt.overleaf.com/learn/latex/Inserting_Images}{Inserindo imagens - Overleaf}}

\bibitem{overleafTab} Site Overleaf - Tabelas. {\it Disponível em \href{https://pt.overleaf.com/learn/latex/Tables}{Tabelas - Overleaf}}

\end{thebibliography}


\end{document}
